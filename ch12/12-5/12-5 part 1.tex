\documentclass[12pt]{exam}
    \usepackage[top=1in, bottom=1in, left=.45in, right=.45in]{geometry}
    \usepackage{amsmath,amsthm,amssymb,amstext}
    \usepackage{enumerate,enumitem}
    \usepackage{tikz,float,graphicx}
    \usepackage{microtype}
    \usepackage{bm,tikz}
        \usetikzlibrary{calc}
    \usepackage{multicol}
    \usepackage{nicematrix}
    \usepackage{cleveref}
    \usepackage[framemethod=tikz]{mdframed}
    
    \newcommand{\course}{MTH 234 Summer 2021}
    \newcommand{\qdate}{Equations of lines and planes} %PUT DATE HERE
    \newcommand{\quiz}{Group Work} 
    
    \newcommand{\R}{\mathbb{R}}
    
    \newcommand{\ba}{\bm{a}}
    \newcommand{\bb}{\bm{b}}
    \newcommand{\bc}{\bm{c}}
    \newcommand{\bi}{\bm{i}}
    \newcommand{\bj}{\bm{j}}
    \newcommand{\bk}{\bm{k}}
    \newcommand{\br}{\bm{r}}
    \newcommand{\bv}{\bm{v}}
    \newcommand{\gen}[1]{\left\langle #1 \right\rangle}

\newtheorem*{theorem}{Theorem}
\surroundwithmdframed[]{theorem}

\theoremstyle{definition}
    \newtheorem*{definition}{Definition}
    \surroundwithmdframed[]{definition}
    \newtheorem*{info}{Useful Information}
    \surroundwithmdframed[]{info}
\theoremstyle{remark}
    \newtheorem*{remark}{Remark}
    \surroundwithmdframed[]{remark}
    

%%%%%%%%%%%%%%%%%%%%%%%
% HEADER AND FOOTER
%%%%%%%%%%%%%%%%%%%%%%%
\pagestyle{headandfoot}
\firstpageheadrule
\runningheadrule
\firstpageheader{\course}{\quiz}{\qdate}
\runningheader{\course}{\quiz}{\qdate}
\runningfooter{}{}{}


\usepackage{color}
\shadedsolutions
\definecolor{SolutionColor}{rgb}{0.8,0.9,1}
\printanswers
%\noprintanswers


\begin{document}
\section*{12.5 Part 1: Lines}

A note about notation:
\begin{center}
If \(P(x_0,y_0,z_0)\) is a point then \(\bm{p}\) will denote the vector 
    \(
        \bm{p}=\gen{x_0,y_0,z_0}.
    \)
    \end{center}
\begin{info}
\phantom{.}
    \begin{itemize}
            \item The \textbf{line through a point \(P\) parallel to \(\bv\)} is given by
            \begin{equation}
                \br(t)=\bm{p}+t\bv,\quad t\in \R \label{eqn:point_vector}
            \end{equation}
            \item The \textbf{line segment} beginning at a point \(R_0\) and ending at the point \(R_1\) is given by
            \begin{align}
                \br(t) & = (1-t)\br_0+t\br_1\quad t\in\R\\
                \text{or equivalently}\quad \br(t) & =\br_0+t(\br_1-\br_0)\quad t\in \R\label{eqn:twopoints_alt}
            \end{align}
            % Note that \Cref{eqn:twopoints_alt} is just \Cref{eqn:point_vector} with \(\bv\) equal to the vector \(\br_1-\br_0\).
            \item A \textbf{parametric equation} for a line is of the form
            \[
                \br{t}=\gen{at+x_0,bt+y_0,ct+z_0}
            \]
            and if \(a,b,c\ne 0\) a \textbf{symmetric equation\footnote[1]{
            This comes from $\gen{x,y,z}=\gen{at+x_0,bt+y_0,ct+z_0}$ and solving each component for \(t\), e.g. $x=at+x_0,\ldots$}
            } is given by 
            \[
                \dfrac{x-x_0}{a}=\dfrac{y-y_0}{b}=\dfrac{z-z_0}{c}
            \]

    \end{itemize}
        
\end{info}

\begin{questions}

\question Find an equation for the line segment \(\br(t)\) with \(\br(0)=(3,-7,0)\) and \(\br(1)=(18,-7,\pi)\), \(0\le t \le 1\).

\ifprintanswers
        \begin{solution}
            \begin{align*}
                \br(t) & = \gen{3,-7,0}+t(\gen{18,-7,\pi}-\gen{3,-7,0})\\
                    & = \gen{3,-7,0}+t(\gen{15,0,\pi})\\
                    & = \gen{3+15t,-7,\pi t}
            \end{align*}
        \end{solution}
    \else
        \vfill
    \fi    

\question Let \(\bm{f}(t)=\gen{1+2t,-10t,2200111-7t}\). Let \(L\) be a line parallel to \(\bm{f}(t)\) that passes through the point \((0,0,0)\). 
\begin{parts}
    \part Find a parametric equation for \(L\).

    \ifprintanswers
        \begin{solution}
            From 
            \begin{align*}
                \bm{f}(t) & = \gen{1+2t,-10t,2200111-7t}\\
                    & = \gen{1,0,2200111}+t\gen{2,-10,-7}
            \end{align*}
            We see that \(L\) is a line through \((0,0,0)\) and parallel to the vector \(\gen{2,-10,-7}\) and is given by
            \begin{align*}
                \br(t) & = \gen{0,0,0}+t\gen{2,-10,-7}\\
                    & = \gen{2t,-10t,-7t}.
            \end{align*}
        \end{solution}
    \else
        \vfill
    \fi 

    \part Use the result above to find a symmetric equation for \(L\).

    \ifprintanswers
        \begin{solution}
            \begin{align*}
                x = 2t & \iff \frac{x}{2}=t\\
                y = -10t & \iff \frac{-y}{10}=t\\
                z = -7t & \iff \frac{-z}{7}=t
            \end{align*}
            Which gives 
            \[
                \frac{x}{2} = \frac{y}{-10}=\frac{z}{-7}
            \]
        \end{solution}
    \else
        \vfill
    \fi 

\end{parts}

\question Suppose \(L\) is represented by the equation \(\br_1(t)=\gen{2t,1-t,2\pi+\pi t}\). Determine which of the following points belong to \(L\)
\begin{parts}
\part \(P(-4,3,0)\)
    \ifprintanswers
        \begin{solution}
            If \(P\) is in \(L\) then when \(x=-4\) we have 
            \[
                -4=2t \quad\iff\quad t=-2.
            \]
            When \(t=-2\) the line \(L\) passes through the point 
            \[
                (2(-2),1-(-2),2\pi+(-2)\pi) = (-4,3,0).
            \]
        Which means the line \(L\) passes through \(P\).
        \end{solution}
    \else
        \vfill
    \fi 

\part \(Q(1,1/2,5\pi)\)
    \ifprintanswers
        \begin{solution}
            \(Q\) is not a point on \(L\)
        \end{solution}
    \else
        \vfill
    \fi 

\part \(R(20,-9,12\pi)\)
    \ifprintanswers
        \begin{solution}
            The line \(L\) passes through \(R\) when \(t=10\).
        \end{solution}
    \else
        \vfill
    \fi 
\end{parts}


\question Let \(L\) denote the line with symmetric equation
\[
    \dfrac{x-1}{2}=y=\dfrac{z+1}{3}
\]
\begin{parts}

\part Find a parametric equation representing \(L\).
    \ifprintanswers
        \begin{solution}
            We have 
            \begin{gather*}
                t=\dfrac{x-1}{2} \iff x=2t+1\\
                t=y\\
                t=\dfrac{z+1}{3} \iff z=3t-1
            \end{gather*}
            which gives the parametric equation
            \[\br(t)=\gen{2t+1,t,3t-1}\]
        \end{solution}
    \else
        \vfill
    \fi 

\part Determine if the line that passes through the points \((1,-5,5)\) and \((-1,0,2)\) intersects \(L\). If not, determine if it is parallel or skew to \(L\).
    
    \ifprintanswers
        \begin{solution}
            The line through \((1,-5,5)\) and \((-1,0,2)\) is given by
            \begin{align*}
                \bm{s}(t) & = \gen{-1,0,2} + t*\left(\gen{1,-5,5}-\gen{-1,0,2}\right)\\
                    & = \gen{-1,0,2}+t\gen{2,-5,3}\\
                    & = \gen{2t-1,-5t,3t+2}.
            \end{align*}
            The lines do not intersect (assume \(2t+1=2t-1\) and it follows).
            From 
            \begin{align*}
                \br(t) & = \gen{\phantom{-}1,0,-1} +  t \gen{2,\phantom{-}1,3}\\
                \bm{s}(t) & = \gen{-1,0,\phantom{-}2} + t \gen{2,-5,3}
            \end{align*}
            We see the lines are not parallel since 
            \[
                \frac{2}{2}\ne \frac{1}{-5} \ne \frac{3}{3}
            \]
            which means the lines are skew.
        \end{solution}
    \else
        \vfill
    \fi 

\end{parts}


\end{questions}


\end{document}


%\question Sample question
%
%\ifprintanswers
%        \begin{solution}
%        \end{solution}
%    \else
%        \vfill
%    \fi 