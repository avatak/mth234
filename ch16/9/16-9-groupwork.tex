\documentclass[12pt]{exam}

\newcommand{\course}{MTH 234 Summer 2021}
\newcommand{\qdate}{The Divergence Theorem} %PUT DATE HERE
\newcommand{\quiz}{Group Work} 
\newcommand{\mbf}{\mathbf{F}}
\newcommand{\intfs}{\iint_S~\mbf\cdot d\bm{S}}
\newcommand{\intdiv}{\iiint_E~\mathrm{div}~\mbf~dV}

    \usepackage[top=1in, bottom=1in, left=.45in, right=.45in]{geometry}
    \usepackage{amsmath,amsthm,amssymb,amstext}
    \usepackage{enumerate,enumitem}
    \usepackage{tikz,float,graphicx}
    \usepackage{microtype}
    \usepackage{bm,tikz}
        \usetikzlibrary{calc}
    \usepackage{multicol}
    \usepackage{nicematrix}
    \usepackage{cleveref}
    \usepackage[framemethod=tikz]{mdframed}
    
    %\newcommand{\course}{MTH 234 Summer 2021}
    %\newcommand{\qdate}{Equations of lines and planes} %PUT DATE HERE
    %\newcommand{\quiz}{Group Work} 
    
    \newcommand{\R}{\mathbb{R}}
    
    \newcommand{\ba}{\bm{a}}
    \newcommand{\bb}{\bm{b}}
    \newcommand{\bc}{\bm{c}}
    \newcommand{\bi}{\bm{i}}
    \newcommand{\bj}{\bm{j}}
    \newcommand{\bk}{\bm{k}}
    \newcommand{\br}{\bm{r}}
    \newcommand{\bv}{\bm{v}}
    \newcommand{\gen}[1]{\left\langle #1 \right\rangle}

\newtheorem*{theorem}{Theorem}
\surroundwithmdframed[]{theorem}

\theoremstyle{definition}
    \newtheorem*{definition}{Definition}
    \surroundwithmdframed[]{definition}
    \newtheorem*{info}{Useful Information}
    \surroundwithmdframed[]{info}
\theoremstyle{remark}
    \newtheorem*{remark}{Remark}
    \surroundwithmdframed[]{remark}
    

%%%%%%%%%%%%%%%%%%%%%%%
% HEADER AND FOOTER
%%%%%%%%%%%%%%%%%%%%%%%
\pagestyle{headandfoot}
\firstpageheadrule
\runningheadrule
\firstpageheader{\course}{\quiz}{\qdate}
\runningheader{\course}{\quiz}{\qdate}
\runningfooter{}{}{}


\usepackage{color}
\shadedsolutions
\definecolor{SolutionColor}{rgb}{0.8,0.9,1}

\usepackage{pgfplots}
    \pgfplotsset{every axis/.append style={
                    axis x line=middle,    % put the x axis in the middle
                    axis y line=middle,    % put the y axis in the middle
                    axis line style={<->}, % arrows on the axis
                    xlabel={$x$},          % default put x on x-axis
                    ylabel={$y$},          % default put y on y-axis
                    grid=both,
                    %xtick={-4,...,-1,1,...,3},
                    %ytick={-1,1,}
    }}
    \pgfplotsset{compat=1.17}

\newcommand{\bif}{\quad\iff\quad}

\printanswers
%\noprintanswers

\begin{document}

\section*{\qdate}

\begin{theorem}[The Divergence Theorem]
    Let \(E\) be a simple solid region and let \(S\) be the boundary surface of \(E\), given with positive (outward) orientation. Let \(\mbf\) be a vector field whose component functions have continuous partial derivatives in an open region that contains \(E\). Then
    \[
        \iint_S~\mbf\cdot d\bm{S}=\iiint_E~\mathrm{div}~\mbf~dV
    \]
\end{theorem}



%\subsection*{Template}

\begin{questions}

\question Verify the divergence theorem is true for the vector field \(\mbf{F}(x,y,z)=\gen{z,y,x}\) on \(E\) where \(E\) is the solid ball \(x^2+y^2+z^2\le 16\) with boundary sphere \(S\) defined by \(x^2+y^2+z^2=16\). In other words, evaluate both 
    \[
        \iint_S\mbf\cdot d\bm{S}
    \]
    and
    \[
        \iiint_E~\mathrm{div}~\mbf~dV
    \]
    and show they are equal.
    \ifprintanswers
        \begin{solution}
            Parameterize \(S\) by 
            \[
                \bm{r}(\phi,\theta)=\gen{4\sin\phi\cos\theta,4\sin\phi\sin\theta,4\cos\phi}.
            \]
            Then 
                \[
                    \mbf(\bm{r}(\phi,\theta))=\gen{4\cos\phi,4\sin\phi\sin\theta,4\sin\phi\cos\theta}
                \]
                and 
                \[
                \mbf(\bm{r}(\phi,\theta))\cdot(r_\phi\times r_\theta)=64\left(2\cos\phi\sin^2\phi\cos\theta+\sin^3\phi\sin^2\theta\right)
                \]
            
            
            \begin{align*}
                \iiint_E~\mathrm{div}~\mbf~dV & = \iint_E~1~dV\\
                & = \frac{64}{3}\pi.
            \end{align*}
        \end{solution}
    \else
        \vfill
    \fi

\newpage 

\question Use the divergence theorem (assume all conditions are satisfied) to prove the identity
\[
        \iint_S~\mathrm{curl}~\mbf\cdot~d\mathrm{S}=0.
\]

    \ifprintanswers
        \begin{solution}
            Follows immedietly from the fact that \(\mathrm{div}~\mathrm{curl}~\mbf=0.\)
        \end{solution}
    \else
        \vfill
    \fi

\question Use the divergence theorem to calculate the flux of 
    \(\mbf(x,y,z)=xye^z\bi+x^2z^3\bj-ye^z\bk\)
    across the surface of the box bounded by the coordinate planes and the planes \(x=3,y=2,z=1\).

    \ifprintanswers
        \begin{solution}
            \[
                \mathrm{div}~\bm{F}  = ye^z+0-ye^z=0 \Longrightarrow \iiint_E~\div~\mbf~dV = 0.
            \]
        \end{solution}
    \else
        \vfill
    \fi

\newpage 

\question Let 
    \[
        \mbf(x,y,z)=\gen{z^2x,\frac{1}{3}y^3+\tan z,~(x^2z+y^2)}
    \] 
    and let 
    \(S\) be the top half of the sphere \(x^2+y^2+z^2=1,\) 
    ~\(z\ge 0\). 

    Use the divergence theorem to evaluate \(\intfs\).
    
    \emph{Hint: \(S\) is not a closed surface, but \(S_0\) equals \(S\) along with the disk \(D_0\), \(x^2+y^2\le 1\) is. Use the divegence theorem with \(S_0\) and use \(S=S_0-D_0\).}
    \ifprintanswers
        \begin{solution}
            \[
                \mathrm{div}~\mbf = z^2+y^2+x^2
            \]
            \(S_0\) is \(\{(\rho,\phi,\theta)~|~0\le\rho\le 1,~0\le\phi\le\pi/2,~0\le\theta\le 2\pi\}\).
            \begin{align*}
                \intdiv & = \int_0^{2\pi}\int_0^{\pi/2}\int_0^1(\rho^2)\rho^2\sin\phi~d\rho~d\phi~d\theta\\
                & = 2\pi\left(\int_0^{\pi/2}\sin\phi~d\phi\right)\left(\int_0^1\rho^{4}~d\rho\right)\\
                & = 2\pi\left(2\right)\left(1/5\right)\\
                & = \frac{4\pi}{5}.
            \end{align*}
            
        \end{solution}
    \else
        \vfill
    \fi


\end{questions}

\end{document}

% soln : Question environment
    \ifprintanswers
        \begin{solution}
        \end{solution}
    \else
        \vfill
    \fi