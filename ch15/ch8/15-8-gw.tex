\documentclass[12pt]{exam}

\newcommand{\course}{MTH 234 Summer 2021}
45\newcommand{\qdate}{15.8 Triple Integrals in Cylindrical Coordinates} %PUT DATE HERE
\newcommand{\quiz}{Group Work} 

    \usepackage[top=1in, bottom=1in, left=.45in, right=.45in]{geometry}
    \usepackage{amsmath,amsthm,amssymb,amstext}
    \usepackage{enumerate,enumitem}
    \usepackage{tikz,float,graphicx}
    \usepackage{microtype}
    \usepackage{bm,tikz}
        \usetikzlibrary{calc,positioning}
    \usepackage{multicol}
    \usepackage{nicematrix}
    \usepackage{cleveref}
    \usepackage[framemethod=tikz]{mdframed}
    \usepackage{graphicx}
    \usepackage[export]{adjustbox}
    
    %\newcommand{\course}{MTH 234 Summer 2021}
    %\newcommand{\qdate}{Equations of lines and planes} %PUT DATE HERE
    %\newcommand{\quiz}{Group Work} 
    
    \newcommand{\R}{\mathbb{R}}
    
    \newcommand{\ba}{\bm{a}}
    \newcommand{\bb}{\bm{b}}
    \newcommand{\bc}{\bm{c}}
    \newcommand{\bi}{\bm{i}}
    \newcommand{\bj}{\bm{j}}
    \newcommand{\bk}{\bm{k}}
    \newcommand{\br}{\bm{r}}
    \newcommand{\bv}{\bm{v}}
    \newcommand{\bu}{\bm{u}}
    \newcommand{\gen}[1]{\left\langle #1 \right\rangle}
    \newcommand{\pd}[2]{\dfrac{\partial #1}{\partial #2}}

\newtheorem*{theorem}{Theorem}
\surroundwithmdframed[]{theorem}

\theoremstyle{definition}
    \newtheorem*{definition}{Definition}
    \surroundwithmdframed[]{definition}
    \newtheorem*{info}{Useful Information}
    \surroundwithmdframed[]{info}
\theoremstyle{remark}
    \newtheorem*{remark}{Remark}
    \surroundwithmdframed[]{remark}
    

%%%%%%%%%%%%%%%%%%%%%%%
% HEADER AND FOOTER
%%%%%%%%%%%%%%%%%%%%%%%
\pagestyle{headandfoot}
\firstpageheadrule
\runningheadrule
\firstpageheader{\course}{\quiz}{\qdate}
\runningheader{\course}{\quiz}{\qdate}
\runningfooter{}{}{}


\usepackage{color}
\shadedsolutions
\definecolor{SolutionColor}{rgb}{0.8,0.9,1}

\usepackage{pgfplots}
    \pgfplotsset{every axis/.append style={
                    axis x line=middle,    % put the x axis in the middle
                    axis y line=middle,    % put the y axis in the middle
                    axis z line=middle,
                    axis line style={<->}, % arrows on the axis
                    xlabel={$x$},          % default put x on x-axis
                    ylabel={$y$},          % default put y on y-axis
                    zlabel={$z$},
                    grid=both,
                    %xtick={-4,...,-1,1,...,3},
                    %ytick={-1,1,}
    }}
    \pgfplotsset{compat=1.17}

\newcommand{\bif}{\quad\iff\quad}

\printanswers
%\noprintanswers

\begin{document}

\section*{\qdate}

%\subsection*{Template}

\begin{questions}

\question Sketch or describe the surface given (all coordinates are cylindrical):
\begin{parts}
    \part \(r=5\)
        \ifprintanswers
        \begin{solution}
            A cylinder of radius 5, i.e. a circle of radius 5 in the \(xy\)-plane extended infinitely in both directions of the \(z\)-axis.
        \end{solution}
    \else
        \vfill
    \fi
    \part \(\theta = 0\)
        \ifprintanswers
        \begin{solution}
            The \(xz\)-plane.
        \end{solution}
    \else
        \vfill
    \fi
    \part \(\theta = \pi/4\)
        \ifprintanswers
        \begin{solution}
            A plane parallel to the \(z\)-axis and the line \(y=x\).
        \end{solution}
    \else
        \vfill
    \fi
    \part \(z=4-r^2\)
        \ifprintanswers
        \begin{solution}
            When \(\theta=0\) this is the line \(z=4-x^2\) in the \(xz\)-plane. Since \(\theta\) can vary, the surface is the surface obtained by rotating this line around as seen below
            \begin{center}

            \end{center}
        \end{solution}
    \else
        \vfill
    \fi
    \part \(0\le r\le 2\), \(-\pi/2\le \theta\le \pi/2\), \(0\le z \le 1\).
        \ifprintanswers
        \begin{solution}
            The right half of a circle of radius \(2\) that is \(1\) unit thick.
        \end{solution}
    \else
        \vfill
    \fi
\end{parts}
 
 \question Convert the following (given in rectangular coordinates) to cylindrical coordinates:
 \begin{parts}
    \part The point \((-1,1,1)\)
        \ifprintanswers
        \begin{solution}
            \begin{align*}
                r^2&=(-1)^2+1^2\\
                r &= \sqrt{2}\\
                \tan\theta &= \frac{1}{-1}\\
                    \theta &= \tan^{-1}(-1)\\
                    \theta &= -\pi/4
            \end{align*}
            \[\boxed{(\sqrt{2},-\pi/4,1)}\]
        \end{solution}
    \else
        \vfill
    \fi
    \part The point \((2,-\pi/2,1)\)
        \ifprintanswers
        \begin{solution}
            \[
                r=\sqrt{4+\pi^2/4}
            \]
            \[
                \theta=\tan^{-1}\left(-\frac{\pi}{4}\right)
            \]
            \[
            \boxed{\left(sqrt{4+\pi^2/4}, \tan^{-1}\left(-\frac{\pi}{4}\right), 1\right)}
            \]
        \end{solution}
    \else
        \vfill
    \fi
    \part \(z=x^2-y^2\)
        \ifprintanswers
        \begin{solution}
        
        \begin{align*}
            z &= (r\cos\theta)^2-(r\sin\theta)^2\\
              &= r^2\cos^{2}\theta-r^2\sin^2\theta\\
        \end{align*}
            
            \[\boxed{z=r^2\left(\cos ^2\theta-\sin^2\theta \right)}\]
        \end{solution}
    \else
        \vfill
    \fi
    \part \(x^2-x+y^2+z^2 = 1\)
        \ifprintanswers
        \begin{solution}
            \(z^2=1+r\cos\theta-r^2\)
        \end{solution}
    \else
        \vfill
    \fi
 \end{parts}

 \newpage

\question Sketch the surface whose volume is given by the integral
\[
        \int_{-\pi/2}^{\pi/2}\int_0^2\int_0^{r^2}~r~dz~dr~d\theta.
\]
What is the volume of this surface?
\ifprintanswers
        \begin{solution}
            The volume is given by 
            \begin{align*}
            \int_{-\pi/2}^{\pi/2}\int_0^2\int_0^{r^2}~r~dz~dr~d\theta & = \int_{-\pi/2}^{\pi/2}\int_0^2 r^3~dr~d\theta\\
            &= \int_{-\pi/2}^{\pi/2} 4~d\theta\\
            &= 4\pi
            \end{align*}
        \end{solution}
    \else
        \vfill
    \fi 

 \question Use cylindrical coordinates to evaluate the following:
 \begin{parts}
    \part \(\iiint_E(x+y+z)~dV\) where \(E\) is the solid in the first octant that lies under the parabaloid \(z=4-x^2-y^2\).
        \ifprintanswers
        \begin{solution}
        \end{solution}
    \else
        \vfill
    \fi 

    \part \(\iiint_E\sqrt{x^2+y^2}~dV\) where \(E\) is the region that lies inside the cylinder 
    \(x^2+y^2=16\) and between the planes \(z=-5\) and \(z=4\).
        \ifprintanswers
        \begin{solution}
        \end{solution}
    \else
        \vfill
    \fi
 \end{parts}

 \newpage

 \question Evaluate by changing to cylindrical coordinates:
    \[
        \int_{-2}^{2}\int_{-\sqrt{4-y^2}}^{\sqrt{4-y^2}}\int_{\sqrt{x^2+y^2}}^2~xz~dz~dx~dy.
    \]

\end{questions}

\end{document}

% soln : Question environment
    \ifprintanswers
        \begin{solution}
            Based on the bounds, \(z\) is out

            \begin{align*}
            \int_{-2}^{2}\int_{-\sqrt{4-y^2}}^{\sqrt{4-y^2}}\int_{\sqrt{x^2+y^2}}^2~xz~dz~dx~dy
                & =           
            \end{align*}

        \end{solution}
    \else
        \vfill
    \fi