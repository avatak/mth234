\documentclass[12pt]{exam}

\newcommand{\course}{MTH 234 Summer 2021}
\newcommand{\qdate}{15.4 Double Integrals in Polar Coordinates} %PUT DATE HERE
\newcommand{\quiz}{Group Work} 

    \usepackage[top=1in, bottom=1in, left=.45in, right=.45in]{geometry}
    \usepackage{amsmath,amsthm,amssymb,amstext}
    \usepackage{enumerate,enumitem}
    \usepackage{tikz,float,graphicx}
    \usepackage{microtype}
    \usepackage{bm,tikz}
        \usetikzlibrary{calc}
    \usepackage{multicol}
    \usepackage{nicematrix}
    \usepackage{cleveref}
    \usepackage[framemethod=tikz]{mdframed}
    
    %\newcommand{\course}{MTH 234 Summer 2021}
    %\newcommand{\qdate}{Equations of lines and planes} %PUT DATE HERE
    %\newcommand{\quiz}{Group Work} 
    
    \newcommand{\R}{\mathbb{R}}
    
    \newcommand{\ba}{\bm{a}}
    \newcommand{\bb}{\bm{b}}
    \newcommand{\bc}{\bm{c}}
    \newcommand{\bi}{\bm{i}}
    \newcommand{\bj}{\bm{j}}
    \newcommand{\bk}{\bm{k}}
    \newcommand{\br}{\bm{r}}
    \newcommand{\bv}{\bm{v}}
    \newcommand{\gen}[1]{\left\langle #1 \right\rangle}

\newtheorem*{theorem}{Theorem}
\surroundwithmdframed[]{theorem}

\theoremstyle{definition}
    \newtheorem*{definition}{Definition}
    \surroundwithmdframed[]{definition}
    \newtheorem*{info}{Useful Information}
    \surroundwithmdframed[]{info}
\theoremstyle{remark}
    \newtheorem*{remark}{Remark}
    \surroundwithmdframed[]{remark}
    

%%%%%%%%%%%%%%%%%%%%%%%
% HEADER AND FOOTER
%%%%%%%%%%%%%%%%%%%%%%%
\pagestyle{headandfoot}
\firstpageheadrule
\runningheadrule
\firstpageheader{\course}{\quiz}{\qdate}
\runningheader{\course}{\quiz}{\qdate}
\runningfooter{}{}{}


\usepackage{color}
\shadedsolutions
\definecolor{SolutionColor}{rgb}{0.8,0.9,1}

\usepackage{pgfplots}
    \pgfplotsset{every axis/.append style={
                    axis x line=middle,    % put the x axis in the middle
                    axis y line=middle,    % put the y axis in the middle
                    axis line style={<->}, % arrows on the axis
                    xlabel={$x$},          % default put x on x-axis
                    ylabel={$y$},          % default put y on y-axis
                    grid=both,
                    %xtick={-4,...,-1,1,...,3},
                    %ytick={-1,1,}
    }}
    \pgfplotsset{compat=1.17}

\newcommand{\bif}{\quad\iff\quad}

%\printanswers
\noprintanswers

\begin{document}

\section*{\qdate}

%\subsection*{Template}

\begin{info}
    \[
        \displaystyle\iint_R f(x,y)~dA = \int_\alpha^{\beta}\int_{a}^{b}~f(r\cos\theta,r\sin\theta)~r~dr~d\theta 
    \]
\end{info}

\begin{questions}

\question Find \(\iint_R~\sin(x^2+y^2)~dA\) where \(R\) is the region in the first quadrant between the circles centered at the origin and radii 1 and 3.
\ifprintanswers
        \begin{solution}
            \begin{align*}
            \int_0^{\pi/2}\int_1^3 r\sin(r^2)~dr~d\theta & = \int_0^{\pi/2}-\frac{1}{2}\cos(r^2)|1^3~d\theta\\
                &= \frac{\cos(9)-\cos(1)}{2} \int_0^{\pi/2}~1~d\theta\\
                &= \left(\cos(9)-\cos(1)\right)\frac{\pi}{4}.
            \end{align*}
        \end{solution}
    \else
        \vfill
    \fi

\question Find the area inside one loop of the rose \(r=\cos3\theta\)
\ifprintanswers
        \begin{solution}
            We will find the area inside one loop of the rose \(r=\cos(n\theta)\) where \(n\) is any positive integer.

            \(r=0\) when \(n\theta=\frac{\pi}{2}+k\pi\) where \(k\in \Z\), i.e \(\theta=\frac{\pi}{2n}+\frac{k}{n}\pi\). 
            So one full loop of \(r=\cos(n\theta)\) occurs between \(\theta=-\pi/2n\) and \(\pi/2n\). If \(D\) is the region inside the loop,

            \begin{align*}
                \text{Area}(D) &= \iint_D~dA\\
                    & = \int_{-\pi/2n}^{\pi/2n}\int_{0}^{\cos(n\theta)} r~dr~d\theta\\
                    &= \int_{-\pi/2n}^{\pi/2n} \frac{1}{2}\cos^2(n\theta)~d\theta \\
                    &= \int_{-\pi/2n}^{\pi/2n}\frac{1}{2}\frac{1+\cos(2n\theta)}{2}~d\theta\\
                    &= \frac{1}{4}\int_{-\pi/2n}^{\pi/2n} 1+\cos(2n\theta)~d\theta\\
                    &= \frac{1}{4}\left( \theta+\frac{1}{2n}\sin(2n\theta)|_{-\pi/2n}^{\pi/2n} \right)\\
                    &= \frac{1}{4}\left(
                        \frac{\pi}{2n}+\frac{1}{2n}\sin(\pi)-\left(-\frac{\pi}{2n}+\frac{1}{2n}\sin(-\pi)\right)
                        \right)\\
                        &= \frac{1}{4}\left(\frac{\pi}{n}\right)=\frac{\pi}{4n}
            \end{align*}
            In particular, the area of one leaf of \(r=\cos(3\theta)\) is \(\pi/12\). 
        \end{solution}
    \else
        \vfill
    \fi.


\question Find the volume above the cone \(z=\sqrt{x^2+y^2}\) and below the sphere \(x^2+y^2+z^2=1\).
\ifprintanswers
        \begin{solution}
            \[z=\sqrt{x^2+y^2}=\sqrt{r^2}=r\]
            and 
            \[
                x^2+y^2+z^2=1 \quad\Rightarrow\quad z=\sqrt{1-r^2}
            \]
            gives
            \begin{align*}
                \int_0^{2\pi}\int_0^{1/4} r\sqrt{1-r^2}-r^2~dr~d\theta & = 
            \end{align*}

        \end{solution}
    \else
        \vfill
    \fi

\newpage

\question Evaluate \(\int_{-3}^3\int_{0}^{\sqrt{9-x^2}}\sin\left(x^2+y^2\right)~dy~dx\) by converting to polor coordinates.
\ifprintanswers
        \begin{solution}
            Visually, \(y=0\) to \(y=\sqrt{9-x^2}\) is everything between the \(x-axis\) and a circle of radius 3
            Since \(x\) ranges from \(-3\) to \(3\) this is the upper half of the above mentioned circle.
            \begin{align*}
                \int_0^{\pi}\int_{0}^{3} r\sin(r^2)~dr~d\theta & = \int_0^\pi -\frac{1}{2}\cos(r^2)|_0^3 ~d\theta\\
                    &= \frac{1}{2}\left(1-\cos 9\right)\int_{0}^{\pi}1~d\theta\\
                    &= \frac{\pi}{2}(1-\cos(9)).
            \end{align*}
        \end{solution}
    \else
        \vfill
    \fi

\question Use polar coordinates to combine the sum
\[
        \int_{1/\sqrt{2}}^{1}\int_{\sqrt{1-x^2}}^{x}~xy~dy~dx + \int_1^{\sqrt{2}}\int_0^x xy~dy~dx+\int_{\sqrt{2}}^{2}\int_0^{\sqrt{4-x^2}}xy~dy~dx
\] into a single double integral. Then evaluate the integral.
\ifprintanswers
        \begin{solution}
            \begin{center}
                \begin{tikzpi}
            \end{center}
        \end{solution}
    \else
        \vfill
    \fi







\end{questions}

\end{document}

% soln : Question environment
    \ifprintanswers
        \begin{solution}
        \end{solution}
    \else
        \vfill
    \fi