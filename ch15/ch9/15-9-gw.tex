\documentclass[12pt]{exam}

\newcommand{\course}{MTH 234 Summer 2021}
\newcommand{\qdate}{15.9 Spherical Coord. \& 16.5 Curl/Divergence} %PUT DATE HERE
\newcommand{\quiz}{Group Work} 

    \usepackage[top=1in, bottom=1in, left=.45in, right=.45in]{geometry}
    \usepackage{amsmath,amsthm,amssymb,amstext}
    \usepackage{enumerate,enumitem}
    \usepackage{tikz,float,graphicx}
    \usepackage{microtype}
    \usepackage{bm,tikz}
        \usetikzlibrary{calc}
    \usepackage{multicol}
    \usepackage{nicematrix}
    \usepackage{cleveref}
    \usepackage[framemethod=tikz]{mdframed}
    
    %\newcommand{\course}{MTH 234 Summer 2021}
    %\newcommand{\qdate}{Equations of lines and planes} %PUT DATE HERE
    %\newcommand{\quiz}{Group Work} 
    
    \newcommand{\R}{\mathbb{R}}
    
    \newcommand{\ba}{\bm{a}}
    \newcommand{\bb}{\bm{b}}
    \newcommand{\bc}{\bm{c}}
    \newcommand{\bi}{\bm{i}}
    \newcommand{\bj}{\bm{j}}
    \newcommand{\bk}{\bm{k}}
    \newcommand{\br}{\bm{r}}
    \newcommand{\bv}{\bm{v}}
    \newcommand{\gen}[1]{\left\langle #1 \right\rangle}

\newtheorem*{theorem}{Theorem}
\surroundwithmdframed[]{theorem}

\theoremstyle{definition}
    \newtheorem*{definition}{Definition}
    \surroundwithmdframed[]{definition}
    \newtheorem*{info}{Useful Information}
    \surroundwithmdframed[]{info}
\theoremstyle{remark}
    \newtheorem*{remark}{Remark}
    \surroundwithmdframed[]{remark}
    

%%%%%%%%%%%%%%%%%%%%%%%
% HEADER AND FOOTER
%%%%%%%%%%%%%%%%%%%%%%%
\pagestyle{headandfoot}
\firstpageheadrule
\runningheadrule
\firstpageheader{\course}{\quiz}{\qdate}
\runningheader{\course}{\quiz}{\qdate}
\runningfooter{}{}{}


\usepackage{color}
\shadedsolutions
\definecolor{SolutionColor}{rgb}{0.8,0.9,1}

\usepackage{pgfplots}
    \pgfplotsset{every axis/.append style={
                    axis x line=middle,    % put the x axis in the middle
                    axis y line=middle,    % put the y axis in the middle
                    axis line style={<->}, % arrows on the axis
                    xlabel={$x$},          % default put x on x-axis
                    ylabel={$y$},          % default put y on y-axis
                    grid=both,
                    %xtick={-4,...,-1,1,...,3},
                    %ytick={-1,1,}
    }}
    \pgfplotsset{compat=1.17}

\newcommand{\bif}{\quad\iff\quad}

%\printanswers
\noprintanswers

\begin{document}

\section*{\qdate}

\subsection*{Spherical Coordinates}
\[
    \boxed{x=\rho\sin\phi\cos\theta,\quad y=\rho\sin\phi\sin\theta,\quad z=\rho\cos\phi}
\]
\[
    \boxed{\rho^2=x^2+y^2+z^2}
\]

\begin{questions}

\question Convert the following, given in rectangular coordinates, to spherical coordinates
\begin{parts}
    \part The point \((0,-2,0)\)
    \ifprintanswers
        \begin{solution}
            \begin{align*}
                \rho &= \sqrt{0^2+(-2)^2+0^2} = 2\\
                \cos\phi &= z/\rho\\
                    &= 0.
                \phi &= \cos^{-1}(0)\\
                    &= \pi/2\\
                \cos\theta &= x/\rho\sin\phi \\
                    & = 0\\
                    \theta & = 3\pi/2 (\text{y<0})
            \end{align*}
            \[
            \boxed{2,3\pi/2,\pi/2}
            \]
        \end{solution}
    \else
        \vfill
    \fi
    \part \(z^2=x^2+y^2\)
    \ifprintanswers
        \begin{solution}
            \[
                \boxed{\cos^2\phi=\sin^2\phi}
            \]
        \end{solution}
    \else
        \vfill
    \fi
    \part \(x^2+z^2=9\)
    \ifprintanswers
        \begin{solution}
            \[
                \boxed{\rho^2\left(\sin^2\phi\cos^2\theta+\cos^2\phi\right)=9}
            \]
        \end{solution}
    \else
        \vfill
    \fi
\end{parts}

\question Describe, sketch, or identify the surface given in spherical coordinates
\begin{parts}
    \part \(\phi=\pi/3\)
    \ifprintanswers
        \begin{solution}
            This is the upper half of a cone
        \end{solution}
    \else
        \vfill
    \fi
    \part \(\rho=2\)
    \ifprintanswers
        \begin{solution}
            A sphere of radius 2.
        \end{solution}
    \else
        \vfill
    \fi
    \part \(\theta=\pi/2\)
    \ifprintanswers
        \begin{solution}
            Since we assume \(\rho\ge0\), this is the half of the \(yz\)-plane with \(y\ge 0\).
        \end{solution}
    \else
        \vfill
    \fi
    \part \(\rho=\sin\theta\sin\phi\) (hint: try converting to rectangular coordinates using \(\rho^2\))
    \ifprintanswers
        \begin{solution}
            \begin{align*}
                \rho & =\sin\theta\sin\phi\\
                \rho^2&= \rho\sin\theta\sin\phi\\
                x^2+y^2+z^2 = y\\
                x^2+y^2-y +(-1/2)^2 +z^2 = (-1/2)^2\\
                x^2+\left(y-1/2\right)^2+z^2=1/4
            \end{align*}
            This is a sphere of radius \(1/2\) centered at \(0,1/2,0\).
        \end{solution}
    \else
        \vfill
    \fi
\end{parts}

\newpage

\question Evaluate \(\iiint_B(x^2+y^2+z^2)^2~dV\) where \(B\) is a ball centered at the origin with radius \(5\).
\ifprintanswers
        \begin{solution}
        \begin{align*}
            \iiint_B(x^2+y^2+z^2)^2~dV &= \int_0^{\pi}\int_0^{2\pi}\int_0^5 (\rho^2)^2\rho^2\sin\phi~d\rho~d\theta~d\phi\\
            & = \int_0^{\pi}\sin\phi\int_0^{2\pi}\int_0^5 \rho^6~d\rho~d\theta~d\phi\\
            & = \int_0^{\pi}\sin\phi\int_0^{2\pi} \frac{5^7}{7}~d\theta~d\phi
            & = \int_0^\pi \sin\phi \frac{5^7(2\pi)}{7} ~d\phi\\
            & = \frac{5^6(2\pi)}{7}\left(\cos(\pi)-\cos(0)\right)\\
            & = -\frac{4\pi5^7}{7}\\
            & = -\frac{312500\pi}{7}
        \end{align*}
        \end{solution}
    \else
        \vfill
    \fi

\question Find the volume of the solid that lies above the cone \(\phi=\pi/3\) and below the sphere \(\rho=4\cos\phi\).
\ifprintanswers
        \begin{solution}
            \[\int_0^{\pi/3}\int_0^{2\pi}\int_0^{4\cos\phi}\rho^2\sin\phi~d\rho~d\theta~d\phi=10\pi\]
        \end{solution}
    \else
        \vfill
    \fi

\newpage

\subsection*{Curl and Divergence}

If \(\bm{F}=\gen{P,Q,R}\),
\[\boxed{
    \mathrm{curl}\bm{F}=\nabla\times \bm{F}=\left|
    \begin{NiceMatrix}
    \bi & \bj & \bk \\
    \pd{}{x} & \pd{}{y} & \pd{}{z}\\
    P & Q & R
    \end{NiceMatrix}\right|}
\]
\[
    \boxed{\mathrm{div}\bm{F} = \nabla\cdot\bm{F}=\pd{P}{x}+\pd{Q}{y}+\pd{R}{z}}
\]

\question Find the curl and divergence of each vector field
\begin{parts}
    \part \(\bm{F}(x,y,z)=(x+yz)\bi+(y+xz)\bj+(z+xy)\bk\)
    \ifprintanswers
        \begin{solution}
            \[
                \mathrm{curl}\bm{F}(x,y,z)= \left|\begin{NiceMatrix}
                    \bi & \bj & \bk\\
                    \pd{}{x} & \pd{}{y} & \pd{}{z}\\
                    x+yz & y+xz & z+xy
                \right|\end{NiceMatrix} = \gen{x-x,-(y-y),z-z} = \bm{0}.
            \]
            \[
                \mathrm{div}\bm{F}= \pd{}{x}\left(x+yz\right)+\pd{}{y}(y+xz)+\pd{}{z}(z+xy)=3.
            \]
        \end{solution}
    \else
        \vfill
    \fi
    \part \(\bm{F}(x,y,z)=\gen{e^x\sin y,e^y\sin z,e^z\sin x}\)
    \ifprintanswers
        \begin{solution}
        \end{solution}
    \else
        \vfill
    \fi
\end{parts}


\question Determine if the given vector field is conservative. If it is, find a function \(f\) so that \(\bm{F}=\nabla f\).
\begin{parts}
    \part \(\bm{F}(x,y,z)=\gen{y^2z^3,2xyz^3,3xy^2z^2}\)
    \ifprintanswers
        \begin{solution}
            \[
                \mathrm{curl}\bm{F}=\gen{6xyz^2-6xyz^2,3y^2z^2-3y^2z^2,2yz^3-2yz^3}=\bm{0}.
            \]
            \begin{align*}
                \int y^2z^3~dx &= xy^2z^3+C\\
                \int 2xyz^3~dy &= xy^2z^3+C\\
                \int 3xy^2z^2~dz &= xy^2z^3+C
            \end{align*}
            \[
                f(x) = xy^2z^3+C
            \]
        \end{solution}
    \else
        \vfill
    \fi
    \part \(\bm{F}(x,y,z) = \gen{3xy^2z^2,2x^2yz^3,3x^2y^2z^2}\)
    \ifprintanswers
        \begin{solution}
            Not conservative.
        \end{solution}
    \else
        \vfill
    \fi
    \part \(\bm{F}(x,y,z)=\bi+\sin z\bj+y\cos z\bk\)
    \ifprintanswers
        \begin{solution}
            Conservative, \(f(x,y,z)=x+y\sin(z)+C\)
        \end{solution}
    \else
        \vfill
    \fi
\end{parts}

\newpage

\question The vector field below is shown in the \(xy\)-plane and looks the same in every parallel plane, i.e. every plane of the form \(z=k\). 
        \begin{center}
                \begin{tikzpicture}[scale=.9]
        \draw[thin,<->] (-.5,0)--(4,0) node[right] {$x$};;
        \draw[thin,<->] (0,-.5)--(0,3) node[above] {$y$};

        \foreach \x in {.5,1,1.5,2,2.5,3,3.5}
        {
                \draw[->,blue] (\x,.25) -- ++(0,.75);
                \draw[->,blue] (\x,1.25)-- ++(0,.5);
                \draw[->,blue] (\x,2)-- ++(0,.4);
                \draw[->,blue] (\x,2.75)-- ++(0,.25);
        }
    \end{tikzpicture}
        \end{center}
\begin{parts}
\part Is \(\mathrm{div}\bm{F}\) positive, negative, or zero?
\ifprintanswers
        \begin{solution}
            Let \(\bm{F}=\gen{P,Q,R}\) as usual. Then 
            \[
                \mathrm{div}\bm{F}=\pd{P}{x}+\pd{Q}{y}+\pd{R}{z}.
            \]
            Based on the image and description, \(\pd{P}{x}=\pd{R}{z}=0\). 
            As \(y\) increases, the magnitude of the vector field in the \(y\) direction is decreasing. Thus \(\pd{Q}{y}<0\) and 
            \[
                \boxed{\mathrm{div}\bm{F}<0}.
            \]
        \end{solution}
    \else
        \vfill
    \fi
\part Explain why \(\mathrm{curl}\bm{F}=0\).
\ifprintanswers
        \begin{solution}
            \[
                \mathrm{curl}\bm{F}=\nabla\times \bm{F}=\left|
    \begin{NiceMatrix}
    \bi & \bj & \bk \\
    \pd{}{x} & \pd{}{y} & \pd{}{z}\\
    P & Q & R
    \end{NiceMatrix}\right|=\gen{\pd{R}{y}-\pd{Q}{z},\pd{P}{z}-\pd{R}{x},\pd{Q}{x}-\pd{P}{y}}
            \]
            Each partial derivative in the above is zero
        \end{solution}
    \else
        \vfill
    \fi 
    \end{parts}

\end{questions}

\end{document}

% soln : Question environment
    \ifprintanswers
        \begin{solution}
        \end{solution}
    \else
        \vfill
    \fi