\documentclass[12pt]{exam}

\newcommand{\course}{MTH 234 Summer 2021}
\newcommand{\qdate}{14.4 Group Work} %PUT DATE HERE
\newcommand{\quiz}{Group Work} 

    \usepackage[top=1in, bottom=1in, left=.45in, right=.45in]{geometry}
    \usepackage{amsmath,amsthm,amssymb,amstext}
    \usepackage{enumerate,enumitem}
    \usepackage{tikz,float,graphicx}
    \usepackage{microtype}
    \usepackage{bm,tikz}
        \usetikzlibrary{calc,positioning}
    \usepackage{multicol}
    \usepackage{nicematrix}
    \usepackage{cleveref}
    \usepackage[framemethod=tikz]{mdframed}
    \usepackage{graphicx}
    \usepackage[export]{adjustbox}
    
    %\newcommand{\course}{MTH 234 Summer 2021}
    %\newcommand{\qdate}{Equations of lines and planes} %PUT DATE HERE
    %\newcommand{\quiz}{Group Work} 
    
    \newcommand{\R}{\mathbb{R}}
    
    \newcommand{\ba}{\bm{a}}
    \newcommand{\bb}{\bm{b}}
    \newcommand{\bc}{\bm{c}}
    \newcommand{\bi}{\bm{i}}
    \newcommand{\bj}{\bm{j}}
    \newcommand{\bk}{\bm{k}}
    \newcommand{\br}{\bm{r}}
    \newcommand{\bv}{\bm{v}}
    \newcommand{\bu}{\bm{u}}
    \newcommand{\gen}[1]{\left\langle #1 \right\rangle}
    \newcommand{\pd}[2]{\dfrac{\partial #1}{\partial #2}}

\newtheorem*{theorem}{Theorem}
\surroundwithmdframed[]{theorem}

\theoremstyle{definition}
    \newtheorem*{definition}{Definition}
    \surroundwithmdframed[]{definition}
    \newtheorem*{info}{Useful Information}
    \surroundwithmdframed[]{info}
\theoremstyle{remark}
    \newtheorem*{remark}{Remark}
    \surroundwithmdframed[]{remark}
    

%%%%%%%%%%%%%%%%%%%%%%%
% HEADER AND FOOTER
%%%%%%%%%%%%%%%%%%%%%%%
\pagestyle{headandfoot}
\firstpageheadrule
\runningheadrule
\firstpageheader{\course}{\quiz}{\qdate}
\runningheader{\course}{\quiz}{\qdate}
\runningfooter{}{}{}


\usepackage{color}
\shadedsolutions
\definecolor{SolutionColor}{rgb}{0.8,0.9,1}

\usepackage{pgfplots}
    \pgfplotsset{every axis/.append style={
                    axis x line=middle,    % put the x axis in the middle
                    axis y line=middle,    % put the y axis in the middle
                    axis z line=middle,
                    axis line style={<->}, % arrows on the axis
                    xlabel={$x$},          % default put x on x-axis
                    ylabel={$y$},          % default put y on y-axis
                    zlabel={$z$},
                    grid=both,
                    %xtick={-4,...,-1,1,...,3},
                    %ytick={-1,1,}
    }}
    \pgfplotsset{compat=1.17}

\newcommand{\bif}{\quad\iff\quad}

\printanswers
%\noprintanswers

\begin{document}

\section*{\qdate}


\subsection*{14.4 - Tangent Planes and Linear Approximations}


\begin{questions}

\question Find the equation to the tangent plane of \(z=x^2+xy+3y^2\) at the point \(1,1,5\).
    \ifprintanswers
        \begin{solution}
            \begin{align*}
                \frac{\partial z}{\partial x}(x,y) & = 2x+y\\
                \frac{\partial z}{\partial x}(1,1) & = 3\\
                \frac{\partial z}{\partial y}(x,y) & = x+6y\\
                \frac{\partial z}{\partial y}(1,1) & = 7\\
            \end{align*}
            Which gives the equation 
            \[
                z-5=3(x-1)+7(y-1)
            \]
            or equivalently 
            \[
                z=3x+7y-5
            \]
        \end{solution}
    \else
        \vfill
    \fi

\question Find the linearization to \(f(x,y)=x^3y^4\) at the point \((1,1)\) and use it to approximate the value of \((1.01)^3(.9)^4\).
\ifprintanswers
        \begin{solution}
            \begin{align*}
                \frac{\partial f}{\partial x}(x,y) & = 3x^2y^4\\
                \frac{\partial f}{\partial x}(1,1) & = 3\\
                \frac{\partial f}{\partial y}(x,y) & = 4x^3y^3\\
                \frac{\partial f}{\partial xy}(1,1) & = 4\\
            \end{align*}
            So the linearization is given by 
            \begin{align*}
                L(x,y) & = f(1,1)+f_{x}(1,1)(x-1)+f_{y}(1,1)(y-1)\\ 
                    & = 1+3(x-1)+4(y-1)\\
                    & = 3x+4y-6
            \end{align*}
            Then we can approximate
            \begin{align*}
                (1.01)^3(0.9)^4 & \approxeq L(1.01,0.9)\\
                    & = 3(1.01)+4(.9)-6\\
                    & = 3.03+3.6-6 \\
                    & = 0.63
            \end{align*}
            If you use an expensive calculator to estimate the value you get 
            \( (1.01)^3(0.9)^4\approxeq 0.6300000000000008\).
        \end{solution}
    \else
        \vfill
    \fi

\question Find the linearization to 
    \[
        f(x,y)=\sqrt{x+y}+(y)^4
    \] at the point \((3,1)\) and use this to estimate the value of
        \(\sqrt{4.25}+(.75)^4\).
\ifprintanswers
        \begin{solution}
            \begin{align*}
                f(1,1) & = \sqrt{3+1}+(1)^4 = 2+1=3\\
                f_{x}(x,y) & = \dfrac{1}{2\sqrt{x+y}}\cdot\frac{\partial f}{\partial x}(x+y) = \dfrac{1}{2\sqrt{x+y}}\\
                f_{x}(3,1) & = \dfrac{1}{2\sqrt{3+1}} = \dfrac{1}{4} = 0.25\\
                f_{y}(x,y) & = \dfrac{1}{2\sqrt{x+y}}+4y^3\\
                f_{y}(3,1) & = \dfrac{1}{2\sqrt{3+1}} = \dfrac{1}{4}+4 =4.25\\
            \end{align*}
            \begin{align*}
                L(x,y) & = f(1,1)+f_{x}(3,1)(x-3)+f_{y}(3,1)(y-1)\\
                    & = 3+(0.25)(x-3)+(4.25)(y-1)
            \end{align*}
            So an estimate is given when \(y=.75\) which required \(x+.75=4.25\Rightarrow x=3.5\) and 
            \begin{align*}
                \sqrt{4.25}+(0.75)^4 & \approx 3+(.25)(3.5-3)+(4.25)(0.75-1)\\
                    & = 3+(0.25)(.5)+(4.25)(-.25)\\
                    & = 3+0.125-1.0625 = 2.0625
            \end{align*}
            The estimate given by a calculator is \(2.378\). The estimate is not as good as the one above because we increased the distance from the estimated value and the point used in our linearization.

        \end{solution}
    \else
        \vfill
    \fi

\question If \(f(x,y)\) is differentiable with \(f(2,5)=6\), \(f_x(2,5)=1\), and \(f_y(2,5)=-1\), estimate the value of \(f(2.2,4.9)\).
\ifprintanswers
        \begin{solution}
            \begin{align*}
                f(2.2,4.9) & \approx f(2,5)+f_{x}(2,5)(2.2-2)+f_{y}(2,5)(4.9-5)\\
                    & = 6+1(0.2)+(-1)(-0.1)\\
                    & = 6.3
            \end{align*}
        \end{solution}
    \else
        \vfill
    \fi

\question 
    
\end{questions}

\end{document} 

%soln : Question environment
    \ifprintanswers
        \begin{solution}
        \end{solution}
    \else
        \vfill
    \fi