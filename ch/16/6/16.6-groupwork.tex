\documentclass[12pt]{exam}

\newcommand{\course}{MTH 234 Summer 2021}
\newcommand{\qdate}{16.6} %PUT DATE HERE
\newcommand{\quiz}{Group Work} 

    \usepackage[top=1in, bottom=1in, left=.45in, right=.45in]{geometry}
    \usepackage{amsmath,amsthm,amssymb,amstext}
    \usepackage{enumerate,enumitem}
    \usepackage{tikz,float,graphicx}
    \usepackage{microtype}
    \usepackage{bm,tikz}
        \usetikzlibrary{calc}
    \usepackage{multicol}
    \usepackage{nicematrix}
    \usepackage{cleveref}
    \usepackage[framemethod=tikz]{mdframed}
    
    %\newcommand{\course}{MTH 234 Summer 2021}
    %\newcommand{\qdate}{Equations of lines and planes} %PUT DATE HERE
    %\newcommand{\quiz}{Group Work} 
    
    \newcommand{\R}{\mathbb{R}}
    
    \newcommand{\ba}{\bm{a}}
    \newcommand{\bb}{\bm{b}}
    \newcommand{\bc}{\bm{c}}
    \newcommand{\bi}{\bm{i}}
    \newcommand{\bj}{\bm{j}}
    \newcommand{\bk}{\bm{k}}
    \newcommand{\br}{\bm{r}}
    \newcommand{\bv}{\bm{v}}
    \newcommand{\gen}[1]{\left\langle #1 \right\rangle}

\newtheorem*{theorem}{Theorem}
\surroundwithmdframed[]{theorem}

\theoremstyle{definition}
    \newtheorem*{definition}{Definition}
    \surroundwithmdframed[]{definition}
    \newtheorem*{info}{Useful Information}
    \surroundwithmdframed[]{info}
\theoremstyle{remark}
    \newtheorem*{remark}{Remark}
    \surroundwithmdframed[]{remark}
    

%%%%%%%%%%%%%%%%%%%%%%%
% HEADER AND FOOTER
%%%%%%%%%%%%%%%%%%%%%%%
\pagestyle{headandfoot}
\firstpageheadrule
\runningheadrule
\firstpageheader{\course}{\quiz}{\qdate}
\runningheader{\course}{\quiz}{\qdate}
\runningfooter{}{}{}


\usepackage{color}
\shadedsolutions
\definecolor{SolutionColor}{rgb}{0.8,0.9,1}

\usepackage{pgfplots}
    \pgfplotsset{every axis/.append style={
                    axis x line=middle,    % put the x axis in the middle
                    axis y line=middle,    % put the y axis in the middle
                    axis line style={<->}, % arrows on the axis
                    xlabel={$x$},          % default put x on x-axis
                    ylabel={$y$},          % default put y on y-axis
                    grid=both,
                    %xtick={-4,...,-1,1,...,3},
                    %ytick={-1,1,}
    }}
    \pgfplotsset{compat=1.17}

\newcommand{\bif}{\quad\iff\quad}
\newcommand{\LR}[1]{\left( #1 \right)}

\printanswers
%\noprintanswers

\begin{document}

\section*{\qdate}


\subsection*{More parameterization of surfaces}

\begin{questions}

\question Let \(\br(u,v)=\gen{u+v,u-v,u^2-v^2}\)
\begin{parts}
    \part Evaulate \(\br(2,-1)\) and \(\br(-1,2)\).
        \ifprintanswers
            \begin{solution}
                \begin{align*}
                    \br(2,-1) & = (2-1,2-(-1),2^2-(-1)^2)\\
                        & = (1,3,3)\\
                    \br(-1,2) & = (-1+2,-1-2,(-1)^2-2^2)\\
                        & = (1,-3,-3)
                \end{align*}
            \end{solution}
        \else
            \vfill
        \fi
    \part Find \(u,v\) so that \(\br(u,v)=(3,-1,-3)\).
        \ifprintanswers
            \begin{solution}
                If \(u+v=3\) then \(v=3-u\). Then
                \[
                    u-v=-1 \bif u-(3-u)=-1 \bif 2u=2\bif u=1
                \]
                Which implies \(v=3-1=2\).
                To verify \(u=1,v=2\) is correct,
                \[
                    \br(1,2)=(1+2,1-2,(1)^2-(2)^2)=(3,-1,-3).
                \]
            \end{solution}
        \else
            \vfill
        \fi
        


    \part Show that \((0,0,1)\) is not a point on the surface.
        \ifprintanswers
            \begin{solution}
                Since \(u-v=0\), \(u=v\). Then \(u+v=0\) gives \(u+u=0\) which means \(u=0=v\).
                But \(\br(0,0)=(0,0,0)\ne (0,0,1)\).
            \end{solution}
        \else
            \vfill
        \fi

    % \(\br(u,v)=\gen{u+v,u-v,u^2-v^2}\)
    \part Recall that a surface \(\br(u,v)\) is \emph{smooth} if \(\br_u\) and \(\br_v\) are both continuous and 
    \(|\br_u\times\br_v|\) is never \(0\) for \((u,v)\) in the interior of the domain 
    (this means that at any point on the surface, the normal vector to the tangent plane is not \(\bm{0}\)).
    Show that \(\br(u,v)\) is continuous.
        \ifprintanswers
            \begin{solution}
            We first compute \(\br_u\), \(\br_v\), \(\br_u\times\br_v\), and \(|\br_u\times\br_v|\).
            \begin{align*}
            \br_u(u,v) & = \gen{
                                1+0,1-0,2u-0
                               }\\
                        & = \gen{1,1,2u}\\
            \br_v(u,v)  & = \gen{0+1,0-1,0-2v}\\
                        & = \gen{1,-1,-2v}.\\
            \br_u\times\br_v & = 
            \left|\begin{NiceMatrix}
                \bi & \bj & \bk \\
                1 & 1 & 2u \\
                1 & -1 & -2v\\
            \end{NiceMatrix}\right|\\
            & = \gen{2u-2v,2u+2v,-2}\\
            |\br_u\times\br_v| & = \sqrt{(2u-2v)^2+(2u+2v)^2+(-2)^2}\\
                & = \sqrt{
                    4u^2-4uv+4v^2+4u^2+4uv+4v^2+4
                }\\
                & = \sqrt{8u^2+8v^2+4}
            \end{align*}
            It is clear that \(\br_u\)
            and \(\br_v\) are continuous everywhere (constant functions and polynomials). 
            The only way \(|\br_u\times\br_v|=\sqrt{8u^2+8v^2+4}=0\) is if \(8u^2+8v^2+4=0\). Since \(8u^2+8v^2+4\ge4\) for any \((u,v)\) we know \(|\br_u\times\br_v|\) is never 0.

            This means the surface is continuous

            \end{solution}
        \else
            \vfill
        \fi

    \part Find an equation of the tangent plane to the parametric surface \(\br(u,v)\) at the point 
    \(u=1\) and \(v=-1\).
        \ifprintanswers
            \begin{solution}
                A point on the tangent plane is given by 
                \(\br(1,-1)=(0,2,0)\). Using \(\br_u\times\br_v\) found above,
                \begin{align*}
                    \br_u\times\br_v(1,-1) & = \gen{2(1)-2(-1),2(1)+2(-1),-2}\\
                    & = \gen{4,0,-2}
                \end{align*}
                Therefore an equation for the tangent plane is
                \begin{align*}
                    \gen{4,0,-2}\cdot\gen{x-0,y-2,z-0} & = 0\\
                    4(x-0)+0(y-2)-2(z-0) & = 0 \\
                    4x-2z & = 0
                \end{align*}
            \end{solution}
        \else
            \vfill
        \fi


    \end{parts}
    
\end{questions}

\end{document}

%soln : Question environment
    \ifprintanswers
        \begin{solution}
        \end{solution}
    \else
        \vfill
    \fi

