\documentclass[12pt]{exam}
    \usepackage[top=1in, bottom=1in, left=.45in, right=.45in]{geometry}
    \usepackage{amsmath,amsthm,amssymb,amstext}
    \usepackage{enumerate,enumitem}
    \usepackage{tikz,float,graphicx}
    \usepackage{microtype}
    \usepackage{bm,tikz}
        \usetikzlibrary{calc}
    \usepackage{multicol}
    \usepackage{nicematrix}
    \usepackage{cleveref}
    \usepackage[framemethod=tikz]{mdframed}
    
    \newcommand{\course}{MTH 234 Summer 2021}
    \newcommand{\qdate}{Equations of lines and planes} %PUT DATE HERE
    \newcommand{\quiz}{Group Work} 
    
    \newcommand{\R}{\mathbb{R}}
    
    \newcommand{\ba}{\bm{a}}
    \newcommand{\bb}{\bm{b}}
    \newcommand{\bc}{\bm{c}}
    \newcommand{\bi}{\bm{i}}
    \newcommand{\bj}{\bm{j}}
    \newcommand{\bk}{\bm{k}}
    \newcommand{\br}{\bm{r}}
    \newcommand{\bv}{\bm{v}}
    \newcommand{\gen}[1]{\left\langle #1 \right\rangle}

\newtheorem*{theorem}{Theorem}
\surroundwithmdframed[]{theorem}

\theoremstyle{definition}
    \newtheorem*{definition}{Definition}
    \surroundwithmdframed[]{definition}
    \newtheorem*{info}{Useful Information}
    \surroundwithmdframed[]{info}
\theoremstyle{remark}
    \newtheorem*{remark}{Remark}
    \surroundwithmdframed[]{remark}
    

%%%%%%%%%%%%%%%%%%%%%%%
% HEADER AND FOOTER
%%%%%%%%%%%%%%%%%%%%%%%
\pagestyle{headandfoot}
\firstpageheadrule
\runningheadrule
\firstpageheader{\course}{\quiz}{\qdate}
\runningheader{\course}{\quiz}{\qdate}
\runningfooter{}{}{}


\usepackage{color}
\shadedsolutions
\definecolor{SolutionColor}{rgb}{0.8,0.9,1}
%\printanswers
\noprintanswers


\begin{document}
\section*{12.5 Part 1: Lines}

A note about notation:
\begin{center}
If \(P(x_0,y_0,z_0)\) is a point then \(\bm{p}\) will denote the vector 
    \(
        \bm{p}=\gen{x_0,y_0,z_0}.
    \)
    \end{center}
\begin{info}
\phantom{.}
    \begin{itemize}
            \item The \textbf{line through a point \(P\) parallel to \(\bv\)} is given by
            \begin{equation}
                \br(t)=\bm{p}+t\bv,\quad t\in \R \label{eqn:point_vector}
            \end{equation}
            \item The \textbf{line segment} beginning at a point \(R_0\) and ending at the point \(R_1\) is given by
            \begin{align}
                \br(t) & = (1-t)\br_0+t\br_1\quad t\in\R\\
                \text{or equivalently}\quad \br(t) & =\br_0+t(\br_1-\br_0)\quad t\in \R\label{eqn:twopoints_alt}
            \end{align}
            % Note that \Cref{eqn:twopoints_alt} is just \Cref{eqn:point_vector} with \(\bv\) equal to the vector \(\br_1-\br_0\).
            \item A \textbf{parametric equation} for a line is of the form
            \[
                \br{t}=\gen{at+x_0,bt+y_0,ct+z_0}
            \]
            and if \(a,b,c\ne 0\) a \textbf{symmetric equation\footnote[1]{
            This comes from $\gen{x,y,z}=\gen{at+x_0,bt+y_0,ct+z_0}$ and solving each component for \(t\), e.g. $x=at+x_0,\ldots$}
            } is given by 
            \[
                \dfrac{x-x_0}{a}=\dfrac{y-y_0}{b}=\dfrac{z-z_0}{c}
            \]

    \end{itemize}
        
\end{info}

\begin{questions}

\question Find an equation for the line segment \(\br(t)\) with \(\br(0)=(3,-7,0)\) and \(\br(1)=(18,-7,\pi)\), \(0\le t \le 1\).

\ifprintanswers
        \begin{solution}
            \begin{align*}
                \br(t) & = \gen{3,-7,0}+t(\gen{18,-7,\pi}-\gen{3,-7,0})\\
                    & = \gen{3,-7,0}+t(\gen{15,0,\pi})\\
                    & = \gen{3+15t,-7,\pi t}
            \end{align*}
        \end{solution}
    \else
        \vfill
    \fi    

\question Let \(\bm{f}(t)=\gen{1+2t,-10t,2200111-7t}\). Let \(L\) be a line parallel to \(\bm{f}(t)\) that passes through the point \((0,0,0)\). 
\begin{parts}
    \part Find a parametric equation for \(L\).

    \ifprintanswers
        \begin{solution}
        \end{solution}
    \else
        \vfill
    \fi 

    \part Use the result above to find a symmetric equation for \(L\).

    \ifprintanswers
        \begin{solution}
        \end{solution}
    \else
        \vfill
    \fi 

\end{parts}

\newpage 

\question Suppose \(L\) is represented by the equation \(\br_1(t)=\gen{2t,1-t,2\pi+\pi t}\). Determine which of the following points belong to \(L\)
\begin{parts}
\part \(P(-4,3,0)\)
    \ifprintanswers
        \begin{solution}
        \end{solution}
    \else
        \vfill
    \fi 

\part \(Q(1,1/2,5\pi)\)
    \ifprintanswers
        \begin{solution}
        \end{solution}
    \else
        \vfill
    \fi 

\part \(Q(20,-9,12\pi)\)
    \ifprintanswers
        \begin{solution}
        \end{solution}
    \else
        \vfill
    \fi 
\end{parts}


\question Let \(L\) denote the line with symmetric equation
\[
    \dfrac{x-1}{2}=y=\dfrac{z+1}{3}
\]
\begin{parts}

\part Find a parametric equation representing \(L\).
    \ifprintanswers
        \begin{solution}
            We have 
            \begin{gather*}
                t=\dfrac{x-1}{2} \iff x=2t+1\\
                t=y\\
                t=\dfrac{z+1}{3} \iff z=3t-1
            \end{gather*}
            which gives the parametric equation
            \[\br(t)=\gen{2t+1,t,3t-1}\]
        \end{solution}
    \else
        \vfill
    \fi 

\part Determine if the line that passes through the points \((1,-5,5)\) and \((-1,0,2)\) intersects \(L\). If not, determine if it is parallel or skew to \(L\).
    
    \ifprintanswers
        \begin{solution}
        \end{solution}
    \else
        \vfill
    \fi 

\end{parts}

\newpage

\section*{12.5 Part 2: Planes}

\begin{info}
    A plane with normal vector \(\bm{n}=\gen{a,b,c}\) containing the point 
        \(P(x_0,y_0,z_0)\) is represented by the equation
        \begin{align*}
              0 & = \bm{n}\cdot(\gen{x,y,z}-\bm{p})\\
                & = \gen{a,b,c}\cdot\gen{x-x_0,y-y_0,z-z_0}\\
                & = ax+by+cz - (ax_0+by_0+cz_0)
        \end{align*}
        or 
        \[
            ax+by+cz=d \quad \text{where} \quad d=ax_0+by_0+cz_0
        \]
    
\end{info}

\question Find a unit normal vector for the plane that contains the line \(\br(t)=\gen{t,-t,2t}\) and the point \(P(0,0,1)\).

\ifprintanswers
        \begin{solution}
        \end{solution}
    \else
        \vfill
    \fi

\question Find the equation of a plane that contains the triangle with vertices \(P(0,0,0)\), \(Q(5,2,0)\), and \(R(0,1,1)\).

\ifprintanswers
        \begin{solution}
        \end{solution}
    \else
        \vfill
    \fi 

\question Find an equation for the plane through the point \((1,-1,-1)\) and parallel to the plane \(5x-y-z=6\).

\ifprintanswers
        \begin{solution}
        \end{solution}
    \else
        \vfill
    \fi

\question At what point does the line through the points \((1,0,1)\) and \((4,-2,2)\) intersect the plane \(x+y+z=6\).

\ifprintanswers
        \begin{solution}
        \end{solution}
    \else
        \vfill
    \fi

\newpage 

\question The planes \(x+y+z=1\) and \(x-2y+3z=1\) intersect in a line.
    \begin{parts}
    \part Find the angle between the two planes (hint: think about normal vectors)
    \ifprintanswers
        \begin{solution}
        \end{solution}
    \else
        \vfill
    \fi 

    \part Find the parametric equation representing the line determined by the intersection of the two planes (hint: think about normal vectors again)
    \ifprintanswers
        \begin{solution}
        \end{solution}
    \else
        \vfill
    \fi 

    \end{parts}

\question Let \(\mathcal{P}\) be the plane represented by 
\[
    x+2y-z=4
\]
Find the equation for the plane that intersects \(\mathcal{P}\) and is parallel to a normal vector for \(\mathcal{P}\).

\ifprintanswers
        \begin{solution}
        \end{solution}
    \else
        \vfill
    \fi 

\newpage

    \question The goal of this exercise is to find the distance from the point \(P(1,-2,4)\) to the plane \(3x+2y+6z=5\).
    \begin{parts}
        \part Find a point \(Q\) on the plane \(3x+2y+6z=5\) (the choice of point does not matter, you just need to choose one).
    \ifprintanswers
        \begin{solution}
        \end{solution}
    \else
        \vfill
    \fi

    \part Find the vector \(PQ\) using the points above.
    \ifprintanswers
        \begin{solution}
        \end{solution}
    \else
        \vfill
    \fi

\begin{remark}
    Unless you were very lucky, the vector \(PQ\) is not the shortest vector connecting \(P\) to the plane. To find the shortest distance possible, we project \(PQ\) onto the normal vector for the plane.
\end{remark}

    \part Find the normal vector \(\bm{n}\) for the plane.
    \ifprintanswers
        \begin{solution}
        \end{solution}
    \else
        \vfill
    \fi

    \part Find \(||\mathrm{proj}_{\bm{n}}(PQ)||\). This is the distance between \(P\) and the plane. See the image below.
    \ifprintanswers
        \begin{solution}
        \end{solution}
    \else
        \vfill
    \fi
    \end{parts}



    \begin{tikzpicture}[scale=2]
        \draw (0,0)--(-3,1)--(-2,2)--(1,1)--(0,0);
        \draw[->] (-1.5,1)--(-1.5,3) node[right] {$\bm{n}$};
        \draw[fill=black] (0,1) circle[radius=.025] node[right] {$Q$};
        \draw[fill=black] (-1,2) circle[radius=.025] node[anchor=south west] {$P$};
        \draw[->] (-1,2)--(-.025,1.025) node[midway,anchor=south west] {$PQ$};
        \draw[very thick,<-] (-1.25,2) -- (-1.25,1) node[anchor=north west] {$\mathrm{proj}_{\bm{n}}(PQ)$};
        \draw[->] (-1,1) to[out=90,in=0] (-1.15,1.5);

    \end{tikzpicture}

\end{questions}


\end{document}


%\question Sample question
%
%\ifprintanswers
%        \begin{solution}
%        \end{solution}
%    \else
%        \vfill
%    \fi 