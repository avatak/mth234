\documentclass[12pt]{exam}

\newcommand{\course}{MTH 234 Summer 2021}
\newcommand{\qdate}{Directional Derivatives} %PUT DATE HERE
\newcommand{\quiz}{Group Work} 

    \usepackage[top=1in, bottom=1in, left=.45in, right=.45in]{geometry}
    \usepackage{amsmath,amsthm,amssymb,amstext}
    \usepackage{enumerate,enumitem}
    \usepackage{tikz,float,graphicx}
    \usepackage{microtype}
    \usepackage{bm,tikz}
        \usetikzlibrary{calc,positioning}
    \usepackage{multicol}
    \usepackage{nicematrix}
    \usepackage{cleveref}
    \usepackage[framemethod=tikz]{mdframed}
    \usepackage{graphicx}
    \usepackage[export]{adjustbox}
    
    %\newcommand{\course}{MTH 234 Summer 2021}
    %\newcommand{\qdate}{Equations of lines and planes} %PUT DATE HERE
    %\newcommand{\quiz}{Group Work} 
    
    \newcommand{\R}{\mathbb{R}}
    
    \newcommand{\ba}{\bm{a}}
    \newcommand{\bb}{\bm{b}}
    \newcommand{\bc}{\bm{c}}
    \newcommand{\bi}{\bm{i}}
    \newcommand{\bj}{\bm{j}}
    \newcommand{\bk}{\bm{k}}
    \newcommand{\br}{\bm{r}}
    \newcommand{\bv}{\bm{v}}
    \newcommand{\bu}{\bm{u}}
    \newcommand{\gen}[1]{\left\langle #1 \right\rangle}
    \newcommand{\pd}[2]{\dfrac{\partial #1}{\partial #2}}

\newtheorem*{theorem}{Theorem}
\surroundwithmdframed[]{theorem}

\theoremstyle{definition}
    \newtheorem*{definition}{Definition}
    \surroundwithmdframed[]{definition}
    \newtheorem*{info}{Useful Information}
    \surroundwithmdframed[]{info}
\theoremstyle{remark}
    \newtheorem*{remark}{Remark}
    \surroundwithmdframed[]{remark}
    

%%%%%%%%%%%%%%%%%%%%%%%
% HEADER AND FOOTER
%%%%%%%%%%%%%%%%%%%%%%%
\pagestyle{headandfoot}
\firstpageheadrule
\runningheadrule
\firstpageheader{\course}{\quiz}{\qdate}
\runningheader{\course}{\quiz}{\qdate}
\runningfooter{}{}{}


\usepackage{color}
\shadedsolutions
\definecolor{SolutionColor}{rgb}{0.8,0.9,1}

\usepackage{pgfplots}
    \pgfplotsset{every axis/.append style={
                    axis x line=middle,    % put the x axis in the middle
                    axis y line=middle,    % put the y axis in the middle
                    axis z line=middle,
                    axis line style={<->}, % arrows on the axis
                    xlabel={$x$},          % default put x on x-axis
                    ylabel={$y$},          % default put y on y-axis
                    zlabel={$z$},
                    grid=both,
                    %xtick={-4,...,-1,1,...,3},
                    %ytick={-1,1,}
    }}
    \pgfplotsset{compat=1.17}

\newcommand{\bif}{\quad\iff\quad}

\printanswers
%\noprintanswers

\begin{document}

\section*{\qdate}


%\subsection*{Template}

\begin{info}
    ~
    For a function \(f(x,y,z)\) the \emph{gradient vector} is 
    \begin{align*}
        \nabla f&= \gen{f_{x},f_{y},f_{z}}\\
            &= \gen{\pd{f}{x},\pd{f}{y},\pd{f}{z}}
    \end{align*}
    and the \emph{directional derivative} in the direction of a \textbf{unit vector} \(u\) is given by
    \begin{align*}
        D_{\bm{u}}f(x,y,z) & = \nabla f(x,y,z)\cdot \bm{u}
    \end{align*}
\end{info}
\begin{questions}

\question Let \(f(x,y)=\sin(2x+3y)\). Find the gradient of \(f\) and evaluate the gradient at the point \(P(-6,4)\). Find the rate of change of \(f\) at \(P\) in the direction of the vector \(u=\gen{\sqrt{3}/2,-1/2}\).
    \ifprintanswers
        \begin{solution}
            \[
                \nabla f = \gen{2\cos(2x+3y),3\cos(2x+3y)}\\
            \]
            So at \(P\) the gradient is 
            \[
                \nabla f(-6,4)=\gen{2\cos(0),3\cos(0)} = \gen{2,3}.
            \]
            Since \(\bu\) is a unit vector, we have 
            \[
                D_{\bu}f(-6,4) = \gen{2,3}\cdot \gen{\sqrt{3}/2,-1/2} = \sqrt{3}-\frac{3}{2}.
            \]
        \end{solution}
    \else
        \vfill
    \fi

\question Find the directional derivative of \(f(x,y)=e^x\cos y\) at the point \((0,0)\) in the direction given by the angle \(\theta=\pi/4\).
    \ifprintanswers
        \begin{solution}
            The unit vector in the direction of \(\theta=\pi/4\) is \(\bu=\gen{\cos(\pi/4),\sin{\pi/4}}=\gen{\sqrt{2}/2,\sqrt{2}/2}\).
            From \(\nabla f=\gen{e^x\cos y,-e^x\sin y}\), 
            \[
                D_{\bu}f(0,0) = \gen{e^0\cos(0),-e^0\sin(0)}\cdot \gen{\sqrt{2}/2,\sqrt{2}/2}= \sqrt{2}/2.
            \]
            \end{solution}
    \else
        \vfill
    \fi

\question Find the gradient of \(f(x,y,z)=x^2yz-xyz^3\). What is the rate of change of \(f\) at the point \(P(2,-1,1)\) in the direction of the point \((2,3,-2)\)?
    \ifprintanswers
        \begin{solution}
            The gradient is 
            \[
                \nabla f=\gen{2xyz-yz^3,x^2z-xz^3,x^2y-3xyz}
            \]
            The direction from \(P\) to \(Q\) is given by \(\gen{2-2,3-(-1),-2-1}=\gen{0,4,-3}\). A unit vector in that direction is \(\gen{0,4/5,-3/5}\).
            So the rate of change is given by
            \[
                \gen{2^2(-1)(1)-(-1)(1)^3,(2)^2(1)-(2)(1),2^2(-1)-3(2)(-1)(1)}\cdot \gen{0,4/5,-3/5} = \gen{-3,2,2}\cdot\gen{0,4/5,-3/5}=\frac{2}{5}.
            \]
        \end{solution}
    \else
        \vfill
        \newpage
    \fi 
\question 
\begin{parts}
    \part How do you find the maximum rate of change for a function \(f\) at a given point \((x_0,y_0,z_0)\)?
        \ifprintanswers
        \begin{solution}
            The maximum rate of change occurs in the direction \(\nabla f\) and is equal to \(|\nabla f|\).
        \end{solution}
    \else
        \vfill
    \fi 
    \part Let \(f(x,y)=xe^y\). Show that at the point \(P(2,0)\), the direction in which \(f\) is increasing the fastest is 
    in the direction \(\gen{1,2}\).
        \ifprintanswers
        \begin{solution}
            \(\nabla f=\gen{e^y,xe^y}\). At \(2,0\) this is \(\gen{1,2}\).
        \end{solution}
    \else
        \vfill
    \fi
    \part What is the largest possible rate of change of \(f\) at the point \(P\)?
        \ifprintanswers
        \begin{solution}
            The largest possible rate of change is \(|\nabla f|=\sqrt{5}\).
        \end{solution}
    \else
        \vfill
    \fi
\end{parts}

\question Recall that for a diffentiable function \(f\) and any unit vector \(\bm{u}\)
\[
    D_uf = \nabla f\cdot \bm{u} = |\nabla f||\bm{u}|| \cos(\theta) = |\nabla f|\cos(\theta)
\] where \(\theta\) is the angle between \(\nabla f\) and \(\bm{u}\).

\begin{parts}
\part Explain why the largest value of \(D_{\bm{u}} f\) occurs when \(\bm{u}\) is in the same direction as \(\nabla f\).
    \ifprintanswers
        \begin{solution}
            Since \(|\nabla f|\ge 0\) and \(-1\le \cos(\theta)\le 1\), the largest value occurs when \(\cos(\theta)=1\), i.e. \(\theta=0\). Which means the rate of change is maximized when the angle between \(\bu\) and \(\nabla f\) is \(0\) (i.e they are the same direction).
        \end{solution}
    \else
        \vfill
    \fi

\part In what direction is the rate of change of \(f\) minimized? Why?
    \ifprintanswers
        \begin{solution}
            Similar to above the rate is minimized \(\cos\theta=-1\) which occurs when \(\theta=\pi\), i.e. when \(\bu\) points in opposite direction of \(\nabla f\).
        \end{solution}
    \else
        \vfill
        \newpage
    \fi 

\end{parts}
\question Find the maximum and minimum rates of change for \(f\) at the given point and the direction in which they occur
\begin{parts}
    \part \(f(x,y)=4y\sqrt{x}\) at \((4,1)\).
        \ifprintanswers
        \begin{solution}
            \(\nabla f=\gen{2y/\sqrt{x},4\sqrt{x}}\), so at \((4,1)\) 
            \(\nabla f(4,1)=\gen{1,8}\).

            The maximum rate of change occurs in the direction \(\gen{1,8}\) and has rate of change equal to \(\sqrt{65}\). The minimum rate of change occurs in the direction \(\gen{-1,-8}\) and has rate of change \(-\sqrt{65}\).
        \end{solution}
    \else
        \vfill
    \fi
    \part \(f(x,y,z)=\sqrt{x^2+y^2+z^2}\) at \((3,6,-2)\).
        \ifprintanswers
        \begin{solution}
            \[\nabla f = \gen{\dfrac{x}{\sqrt{x^2+y^2+z^2}},\dfrac{y}{\sqrt{x^2+y^2+z^2}} },\dfrac{z}{\sqrt{x^2+y^2+z^2}}\]
            and
            \[
                \nabla f(3,6,-2) = \gen{\frac{3}{7},\frac{6}{7},-\frac{2}{7}}
            \]
            Maximum: \(1\) in direction \gen{\frac{3}{7},\frac{6}{7},-\frac{2}{7}},

            Minimum: \(-1\) in direction \gen{-\frac{3}{7},-\frac{6}{7},\frac{2}{7}}
        \end{solution}
    \else
        \vfill
    \fi
\end{parts}

\question Show that at every point on the line \(y=x+1\) the fastest rate of change for \(f(x,y)=x^2+y^2-2x-4y\) is in the direction \(\gen{1,1}\) and that this does not occur at any other point.
        \ifprintanswers
        \begin{solution}
        We need to find all points at which \(\nabla f\) points in the direction of \(\gen{1,1}\). First,
            \begin{align*}
                \nabla f & = \gen{2x-2,2y-4}
            \end{align*}
            Then \(\nabla f\) points in the direction \(\gen{1,1}\) if \(2x-2=2y-4\) which is equivalent to \(y=x+1\).
        \end{solution}
    \else
        \vfill
    \fi

\end{questions}

\end{document}

    \ifprintanswers
        \begin{solution}
        \end{solution}
    \else
        \vfill
    \fioln : Question environment
    \ifprintanswers
        \begin{solution}
        \end{solution}
    \else
        \vfill
    \fi

