\documentclass[12pt]{exam}

\newcommand{\course}{MTH 234 Summer 2021}
\newcommand{\qdate}{14.5 The Chain Rule} %PUT DATE HERE
\newcommand{\quiz}{Group Work} 

    \usepackage[top=1in, bottom=1in, left=.45in, right=.45in]{geometry}
    \usepackage{amsmath,amsthm,amssymb,amstext}
    \usepackage{enumerate,enumitem}
    \usepackage{tikz,float,graphicx}
    \usepackage{microtype}
    \usepackage{bm,tikz}
        \usetikzlibrary{calc,positioning}
    \usepackage{multicol}
    \usepackage{nicematrix}
    \usepackage{cleveref}
    \usepackage[framemethod=tikz]{mdframed}
    \usepackage{graphicx}
    \usepackage[export]{adjustbox}
    
    %\newcommand{\course}{MTH 234 Summer 2021}
    %\newcommand{\qdate}{Equations of lines and planes} %PUT DATE HERE
    %\newcommand{\quiz}{Group Work} 
    
    \newcommand{\R}{\mathbb{R}}
    
    \newcommand{\ba}{\bm{a}}
    \newcommand{\bb}{\bm{b}}
    \newcommand{\bc}{\bm{c}}
    \newcommand{\bi}{\bm{i}}
    \newcommand{\bj}{\bm{j}}
    \newcommand{\bk}{\bm{k}}
    \newcommand{\br}{\bm{r}}
    \newcommand{\bv}{\bm{v}}
    \newcommand{\bu}{\bm{u}}
    \newcommand{\gen}[1]{\left\langle #1 \right\rangle}
    \newcommand{\pd}[2]{\dfrac{\partial #1}{\partial #2}}

\newtheorem*{theorem}{Theorem}
\surroundwithmdframed[]{theorem}

\theoremstyle{definition}
    \newtheorem*{definition}{Definition}
    \surroundwithmdframed[]{definition}
    \newtheorem*{info}{Useful Information}
    \surroundwithmdframed[]{info}
\theoremstyle{remark}
    \newtheorem*{remark}{Remark}
    \surroundwithmdframed[]{remark}
    

%%%%%%%%%%%%%%%%%%%%%%%
% HEADER AND FOOTER
%%%%%%%%%%%%%%%%%%%%%%%
\pagestyle{headandfoot}
\firstpageheadrule
\runningheadrule
\firstpageheader{\course}{\quiz}{\qdate}
\runningheader{\course}{\quiz}{\qdate}
\runningfooter{}{}{}


\usepackage{color}
\shadedsolutions
\definecolor{SolutionColor}{rgb}{0.8,0.9,1}

\usepackage{pgfplots}
    \pgfplotsset{every axis/.append style={
                    axis x line=middle,    % put the x axis in the middle
                    axis y line=middle,    % put the y axis in the middle
                    axis z line=middle,
                    axis line style={<->}, % arrows on the axis
                    xlabel={$x$},          % default put x on x-axis
                    ylabel={$y$},          % default put y on y-axis
                    zlabel={$z$},
                    grid=both,
                    %xtick={-4,...,-1,1,...,3},
                    %ytick={-1,1,}
    }}
    \pgfplotsset{compat=1.17}

\newcommand{\bif}{\quad\iff\quad}
\newcommand{\pd}[2]{\dfrac{\partial #1}{\partial #2}}
\usepackage{cleveref}

\printanswers
%\noprintanswers

\begin{document}

\section*{\qdate}


%\subsection*{Template}



\begin{info}
~

    \noindent 
    Given a differentiable function \(z=f(x,y)\) with \(x=x(t)\) and \(y=y(t)\)
    differentiable functions of \(t\), then \(z\) is a differentiable function of \(t\) with
    \[
        \frac{\partial z}{\partial t}=\pd{f}{x}\pd{x}{t}+\pd{f}{y}\pd{y}{t}
    \]
    or equivalently
    \[
        \pd{z}{t}=\pd{z}{x}\pd{x}{t}+\pd{z}{y}\pd{y}{t}
    \]
    \[
        f_t = f_x x_t + f_y y_t
    \]

    \noindent More generally, if \(z=f(x,y)\) and \(x=x(s,t)\), 
\end{info}

\begin{questions}

\question Suppose \(z=f(x,y)\) where \(f\) is differentiable and
    \begin{equation*}
        \begin{aligned}[c]
            x&=g(t)\\
            g(3)&=2\\
            g'(3)&=5\\
            f_x(2,7)&=6
        \end{aligned}
        \qquad
        \begin{aligned}[c]
            y&=h(t)\\
            h(3)&=7\\
            h'(3)&=-4\\
            f_y(2,7)&=-8
        \end{aligned}
    \end{equation*}
    Find \(\pd{z}{t}\) when \(t=3\).
    \ifprintanswers
        \begin{solution}
            \begin{align*}
                \pd{z}{t} & = \pd{z}{x}\pd{x}{t}+\pd{z}{y}\pd{y}{t}\\
                & = f_{x}g'(t)+f_yh'(t)
                \pd{z}{t}|_{t=3} & = f_{x}(2,7)g'(3)+f_{y}(2,7)h'(3) \\
                    & = 6(5)+(-8)(-4)\\
                    & = 62
            \end{align*}
        \end{solution}
    \else
        \vfill
    \fi

\question Find \(\pd{z}{t}\) if \(z=x^2+y^2+xy\) with \(x=\sin t\) and \(y=e^t\)
    \ifprintanswers
        \begin{solution}

        \[
            (2x+y)\cos t+(2y+x)e^t
        \]
        \end{solution}
    \else
        \vfill
        \newpage
    \fi 
\question Find \(\pd{w}{t}\) if \(w=xe^{y/z}\) with 
\(x=t^2\), \(y=1-t\), \(z=1+2t\).
    \ifprintanswers
        \begin{solution}
            \[
                e^{y/z}\left(2t-(x/z)-(2xy/z^2)\right)
            \]
        \end{solution}
    \else
        \vfill
    \fi

\question Use the chain rule to find 
\(\pd{z}{s}\) and \(\pd{z}{t}\) if \(z=e^r\cos(\theta)\) with \(r=st\) and \(0=\sqrt{s^2+t^2}\)
    \ifprintanswers
        \begin{solution}
            \[
                \pd{z}{s} = e^r\left(t\cos\theta -\dfrac{s}{\sqrt{s^2+t^2}}\sin\theta\right)
            \]
        \end{solution}
    \else
        \vfill
    \fi

\question Let \(z=x^4+x^2y\), with \(x=s+2t-u\) and \(y=stu^2\).
Find \(\pd{z}{s}\), \(\pd{z}{t}\), and \(\pd{z}{u}\) when \(s=4\), \(t=2\), \(u=1\).
    \ifprintanswers
        \begin{solution}
            We will need
            \begin{align*}
                \pd{z}{x} & = 4x^3+2xy\\
                \pd{z}{y}&=x^2\\
                \pd{x}{s}&=1\\
                \pd{y}{s}&=tu^2
            \end{align*}
            When \(s=4,t=2,u=1\), 
            \[
                x=4+2(2)-1 = 7\quad\text{and}\quad y=(4)(2)(1)^2=8
            \]
            Then 
            \begin{align*}
                \pd{z}{s} & = \pd{z}{x}\pd{x}{s}+\pd{z}{y}\pd{y}{s}\\
                    & = \left(4(7)^3+2(7)(8)\right)\left(1\right)+\left(7^2\right)\left((2)(1)^2\right)\\
                        & = 1582
            \end{align*}
            Proceeding similarly gives the following
            \begin{align*}         
                \pd{z}{s} & = 1582\\
                \pd{z}{t} & = 3164\\
                \pd{z}{u} & = -700 
            \end{align*}
        \end{solution}
    \else
        \vfill
    \fi

\newpage

\begin{info}
~
If an equation can be expressed as \(F(x,y)=0\) then 
    \begin{equation} \label{eqn:implicit_xy}
        \pd{y}{x} = -\dfrac{F_x}{F_y}
    \end{equation}
and if \(z=f(x,y)\), then we can rewrite this in the form \(F(x,y,z)=0\) and
    \begin{equation}\label{eqn:implicit_xyz}
        \pd{z}{x} = -\pd{F_x}{F_z} \qquad \pd{z}{y}=-\pd{F_y}{F_z}
    \end{equation}
\end{info}

\question Use \Cref{eqn:implicit_xy} to find \(\pd{y}{x}\)
    \begin{parts}
        \part \(y\cos x=x^2+y^2\)
    \ifprintanswers
        \begin{solution}
            Set \(F(x,y)=x^2+y^2-y\cos x\). Then 
            \[
                F_{x} = 2x+y\sin x \quad\text{and}\quad F_{y}=2y-\cos(x)
            \]
            so 
            \[
                \pd{y}{x} = -\dfrac{2x+y\sin x}{2y-\cos x} = \dfrac{2x+y\sin x}{\cos x-2y}
            \]
        \end{solution}
    \else
        \vfill
    \fi
    \part \(\tan^{-1}(x^2y) = x+xy^2\)
    \ifprintanswers
        \begin{solution}
            Setting \(F(x,y)=\tan^{-1}(x^2y)-x+xy^2 should result in something equivalent to
            \[
                \pd{y}{x} = \dfrac{1+x^4y^2+y^2+x^4y^4-2xy}{x^2-2xy-2x^5y^3}
            \]
        \end{solution}
    \else
        \vfill
    \fi 
    \end{parts}

\question Use \Cref{eqn:implicit_xyz} to find \(\pd{z}{x}\) and \(\pd{z}{y}\)
    \begin{parts}
    \part \(x^2+2y^2+3z^2=1\)
    \ifprintanswers
        \begin{solution}
            If \(F(x,y,z)=x^2+2y^2+3z^2-1\),
            \begin{equation}
            \begin{aligned}
                F_{x}&=2x
                F_{y}&=4y
                F_{z}&=6z
            \end{aligned}
            \end{equation}
            So 
            \[
                \pd{z}{x} = -\dfrac{F_x}{F_z} = -\dfrac{2x}{6z} = -\dfrac{x}{3z}
            \]
            and 
            \[
                \pd{z}{y}= -\dfrac{2y}{3z}
            \]
        \end{solution}
    \else
        \vfill
    \fi
    \part \(e^z=xyz\)
    \ifprintanswers
        \begin{solution}
            Set \(F(x,y,z)=e^z-xyz\).
            \[
                \pd{z}{x}=\dfrac{yz}{e^z-xy},\quad \pd{z}{y}=\dfrac{xz}{e^z-xy}
            \]
        \end{solution}
    \else
        \vfill
    \fi 
    \end{parts}



\end{questions}

\end{document}

%soln : Question environment
    \ifprintanswers
        \begin{solution}
        \end{solution}
    \else
        \vfill
    \fi